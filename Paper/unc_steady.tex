\section{Uncertainty of steady state measurements}
\label{sec:unc_steady}
The training data for compressor maps is obtained from steady state time series data. Steady state time series data does not show any significant monotonic trend in average value but rather shows periodic and random fluctuations (noise) around an average value. The mean value $a$ of the measured variable can be defined as

\begin{equation}
a = \Sigma _{i = 1}^N{\frac{{{a_{mea}}({t_i})}}{N}},
\label{eq:avg_mea}
\end{equation}

where $a_{mea}(t_i)$ is a discrete measurement of the variable at time step $i$, and $N$ is the number of measurements. Measurement devices have an uncertainty which can be considered time independend, often called zero order uncertainty. The limited number of samples in combination with the fluctuations of the variable leads to a second (first order) uncertainty.  Zero-order uncertainty is typically provided by the manufacturer of the measurement device, such as $\pm0.5K$ for T-type thermocouples, or $0.5\%$ of the measured flowrate for Coriolis mass flowmeters. First-order uncertainty is not known in advance but rather needs to be approximated by statistically analyzing the time series data of the steady state measurement. Taylor and Kuyatt (1994)\ref{Kamei:1995} give the overall measurement uncertainty as

\begin{equation}
\Delta a = \sqrt {\Sigma _{i = 1}^N{{\left(\frac{{\Delta {a_{mea}}({t_i})}}{N}\right)}^2} + {{\left(\frac{{{t_{N - 1,1 - \alpha /2}}}}{N}\right)}^2}\frac{{\Sigma _{i = 1}^N{{\left({a_{mea}}({t_i}) - a\right)}^2}}}{{N - 1}}} 
\label{eq:mea_unc_TK}
\end{equation}

where $t_{N-1,1-\alpha}$ is the statistic of the student t distribution with a degree of freedom of N-1 and level at $1-\alpha/2$. The first part of the uncertainty is the sum of squares of the zero-order uncertainty of each observation within the steady state measurement. The second part of the uncertainty is given as the confidence interval of the average value at the level of $1-\alpha/2$. In practice, $\alpha$ is taken as 0.1, and a 95\% confidence interval is usually used in the approximation.