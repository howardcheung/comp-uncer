\section{Linear regression}
\label{sec:linear_regression}
Finding the coefficients of the 10 coefficient maps can be treated as a linear regression problem, with the evaporating and condensing temperature and their linear, squared and cubed combinations being the independent variables ${\vec x}$ and the power consumption of the compressor being the estimated dependent variable $y$. To better understand the uncertainty sources of the model, a brief review of linear regression along with how this applies to the polynomial compressor map is shown subsequently.

\subsection{Review}
\label{sec:linreg_review}
Linear regression is used to estimate the parameters of a linear model for a system that might have slightly nonlinear behavior.  If we had the parameters ${\vec \beta _{true}} $ for the true model that is the best linear model to describe the actual system, we could use the independent inputs to our model (${\vec x^T}$) to calculate the true output $y_{true}$ up to an error $\varepsilon$ caused by the difference between model and system, e.g.

\begin{equation}
{y_{true}} = {\vec x^T}{\vec \beta _{true}} + \varepsilon.
\label{eq:true_lin_model}
\end{equation}

Unfortunately it is not possible to obtain the true model, since it is not possible to obtain the infinite number of input/output data points from the system. Therefore, we need to estimate the model parameters to calculate an estimate of the output. To emphasize the difference between the two models, a hat ( $\hat{}$ ) is used for the estimated values (estimated model parameters $\hat {\vec \beta}$ and estimated model output $\hat y$), and the estimated linear model is

\begin{equation}
\hat y = {\vec x^T}\hat {\vec \beta}.
\label{eq:estimted_lin_model}
\end{equation}

The parameters for the model can be estimated using measurement data. For non-weighted reduction of the squares of the errors, it can be shown that the  parameter vector $\hat {\vec \beta}$ can be calculated as

\begin{equation}
\hat {\vec \beta}  = {({\mathbf{X}}_{train}^T{{\mathbf{X}}_{train}})^{ - 1}}{\mathbf{X}}_{train}^T{\vec y_{train}},
\label{eq:estimte_parameters}
\end{equation}

where the subscript $train$ is used for training data, the training data input matrix ${{\mathbf{X}}_{train}}$, composed of input data vectors ${\vec x}_{train,i}$ for each data point $i$ is constructed as

\begin{equation}
 {{\mathbf{X}}_{train}} = {\left[ {\begin{array}{*{20}{c}} {{{\vec x}_{train,1}}}& \cdots &{{{\vec x}_{train,i}}}& \cdots &{{{\vec x}_{train,n}}}
\end{array}} \right]^T},
\label{eq:input_tr_matrix}
\end{equation}

where $n$ is the total number of data points. The output training data vector is composed of output values $y_{train,i}$ for each data point as

\begin{equation}
{\vec y_{train}} = {\left[ {\begin{array}{*{20}{c}}
  {{y_{train,1}}}& \cdots &{{y_{train,i}}}& \cdots &{{y_{train,n}}} 
\end{array}} \right]^T}.
\label{eq:output_tr_vector}
\end{equation}

The accuracy of the model, measured by the mean sum of square $\sigma$, is 

\begin{equation}
\sigma  = \sqrt {\frac {{\Sigma _i}{{({y_{train,i}} - {{\vec x}^T}\hat {\vec \beta} )}^2}} {n - 1}}.
\label{eq:acc_est_model}
\end{equation}

\subsection{Polynomial compressor map as linear regression problem}
\label{sec:linreg_compmap}
ANSI/AHRI Standard 540 \cite{AHRI:540} estimates the compressor power consumption $\hat {\dot W}$ as

\begin{equation}
\begin{gathered}
  \hat {\dot W} = {\beta _1} + {\beta _2}{T_{evap}} + {\beta _3}{T_{cond}} + {\beta _4}T_{evap}^2 + {\beta _5}{T_{evap}}{T_{cond}} + {\beta _6}T_{cond}^2 + {\beta _7}T_{evap}^3 + \\
  {\beta _8}T_{evap}^2{T_{cond}} + {\beta _9}{T_{evap}}T_{cond}^2   + {\beta _{10}}T_{cond}^3 ,
\end{gathered} 
\label{eq:pwr_map_definition}
\end{equation}

where $\beta_i$ are the (estimated) model coefficients and $T_{evap}$, and $T_{cond}$ are the evaporation and condensing dew point temperatures. Referring to eqn. \ref{eq:estimted_lin_model}, $\hat {\dot W}$ takes the place of $\hat y$, the estimated parameter vector  $\hat {\vec \beta}$ is composed of the $\beta_i$, and the vector of independent variables (here: dew point temperatures), $\vec x$ is constructed as

\begin{equation}
\begin{split}
&\vec x = \\
&{\left[ {\begin{array}{*{20}{c}}
  1&{{T_{evap}}}&{{T_{cond}}}&{T_{evap}^2}&{{T_{evap}}{T_{cond}}}&{T_{cond}^2}&{T_{evap}^3}&{T_{evap}^2{T_{cond}}}&{\begin{array}{*{20}{c}}
  {{T_{evap}}{T_{cond}}}&{T_{cond}^3} .
\end{array}} 
\end{array}} \right]^T}
\end{split}
\label{eq:lin_reg_temp_vec}
\end{equation}
