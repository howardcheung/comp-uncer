\section{Generating experimental results of steady state data with uncertainties from compressor map data}
\label{sec:approx_uncertainty}
In this paper, no real measurement data are taken for training data, and only ideal compressor performance data from the manufacturers without uncertainty is available. In order to test the uncertainty calculation of the compressor map output, a set of training data with uncertainties is approximated from the ideal compressor performance data. If the ideal compressor performance data are the true values of the measurement, its measurement value will be different because it is subjected to the zero-order uncertainty of the measurement device and the first-order uncertainty of the noise of the measurement environment.  Assuming that the uncertainty is the confidence interval of the measurement value at a level of $1-\alpha/2$ and the measurement value follows a normal distribution with a mean value around the true value of the measurement, the standard deviation of the normal distribution will be given by

\begin{equation}
{\sigma _{mea}} = \frac{{\sqrt {{{(\Delta {a_{zero - order}}(a = {a_{true}}))}^2} + {{(\Delta {a_{first - order}}(a = {a_{true}}))}^2}} }}{{{z_{1 - \alpha /2}}}}
\label{eq:std_norm}
\end{equation}

where $z_{1-\alpha/2}$ is the z-normal distribution statistics given at a level of ${1-\alpha/2}$. The observations within steady state can then be approximated by running a random number generator following a normal distribution with mean at the true value of $a$ and standard deviation at $\sigma_{mea}$ multiple times. These values can then be analyzed with the method listed in the section~\ref{sec:unc_steady} to calculate the value and uncertainty of steady state measurement.